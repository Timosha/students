\documentclass[a4paper,12pt]{article}

\usepackage[T2A]{fontenc}
\usepackage[utf8x]{inputenc}
\usepackage[russian,english]{babel}
\usepackage{amssymb,amsfonts,amsmath,cite,enumerate,float,indentfirst}
\usepackage{color}

\usepackage{listings}
\lstloadlanguages{c++,c,lisp,haskell,prolog}
\lstset{
	language=c,inputencoding=utf8x,
	extendedchars=\true,captionpos=b,tabsize=4,
	frame=lines,
	keywordstyle=\color{blue},commentstyle=\color{green},stringstyle=\color{red},
	breaklines=true,showstringspaces=false,basicstyle=\footnotesize
}

\lstset{
	language=haskell,inputencoding=utf8x,
	extendedchars=\true,captionpos=b,tabsize=4,
	frame=lines,
	keywordstyle=\color{blue},commentstyle=\color{green},stringstyle=\color{red},
	breaklines=true,showstringspaces=false,basicstyle=\footnotesize
}
\lstset{
	language=lisp,inputencoding=utf8x,
	extendedchars=\true,captionpos=b,tabsize=4,
	frame=lines,
	keywordstyle=\color{blue},commentstyle=\color{green},stringstyle=\color{red},
	breaklines=true,showstringspaces=false,basicstyle=\footnotesize
}
\lstset{
	language=prolog,inputencoding=utf8x,
	extendedchars=\true,captionpos=b,tabsize=4,
	frame=lines,
	keywordstyle=\color{blue},commentstyle=\color{green},stringstyle=\color{red},
	breaklines=true,showstringspaces=false,basicstyle=\footnotesize
}


\usepackage{geometry} % Меняем поля страницы
\geometry{left=2cm}% левое поле
\geometry{right=1.5cm}% правое поле
\geometry{top=1cm}% верхнее поле
\geometry{bottom=2cm}% нижнее поле

\renewcommand{\theenumi}{\arabic{enumi}}% Меняем везде перечисления на цифра.цифра
\renewcommand{\labelenumi}{\arabic{enumi}}% Меняем везде перечисления на цифра.цифра
\renewcommand{\theenumii}{.\arabic{enumii}}% Меняем везде перечисления на цифра.цифра
\renewcommand{\labelenumii}{\arabic{enumi}.\arabic{enumii}.}% Меняем везде перечисления на цифра.цифра
\renewcommand{\theenumiii}{.\arabic{enumiii}}% Меняем везде перечисления на цифра.цифра
\renewcommand{\labelenumiii}{\arabic{enumi}.\arabic{enumii}.\arabic{enumiii}.}% Меняем везде перечисления 


\title{Лабораторная работа №1}
\author{Кузнецов Д.Б.\and Вагин Д.А.}
\date{09/2010}

\begin{document}

\begin{titlepage}
\newpage

\begin{center}
Федеральное агентство по образованию РФ\\
Пермский Национальный Исследовательский Политехнический Университет\\
Кафедра Информационных Технологий и Автоматизированных Систем
\end{center}

\vspace{8em}

\begin{center}
\Large Кузнецов Д.Б., Вагин Д.А.
\end{center}

\vspace{2em}

\begin{center}
\Huge \textbf{Функциональное и логическое программирование}
\end{center}
\begin{center}
Методические указания по выполниению лабораторных работ\\
для специальностей АСУ и ПОВТ
\end{center}



\vspace{\fill}

\begin{center}
г. Пермь 2011
\end{center}

\end{titlepage}

\newpage
\section{Введение}
Предлагаемый лабораторный практикум предназначен для освоения 
основных принципов программирования на функциональных языках программирования 
LISP и Haskell, и логическом языке программирования Prolog.

\subsection{Требования к аппаратному обеспечению}

\paragraph{Вариант 1: Выполненние работ на локальном компьютере}
Компьютер должен обеспечивать возможность запуска операционной системы, текстового редактора, и средств компиляции исходного кода. Необходимо иметь 200МБ свободного места на жёстком диске для установки необходимого по и хранения исходных текстов программ.

\paragraph{Вариант 2: Выполненние работ на удалённом сервере}
Сервер должен иметь технические характеристики достаточные для запуска операционной системы и сервисов для удалённого подключения. Более точные аппартные характеристики зависят от версии ядра и системных библиотек. Для запуска последних версий без графического режима вполне достаточно процессор с частотой от 300 МГц, ОЗУ от 512 МБ, свободно дисковое пространство от 500 МБ. Локальное рабочее место должно быть оборудовано терминалом, подключенным к серверу.

\subsection{Требования к системному программному обеспечению}
Рассмариваемое ниже системное программное обеспечение должно быть установлено на локальном компьютере (Вариант 1 аппаратного обеспечения) или на сервере (Вариант 2 аппаратного обеспечения).
\paragraph{}
Для выполнения лабораторных работ потребуется инструментарий для сборки компиляции программ на С или С++ (GNU make, GNU gcc, Microsoft Visual Studio, и т.д.), интерпретатор языка LISP (clisp, autolisp, и т.д.), компилятор языка Haskell (ghc), интерпретатор языка Prolog (gprolog). 

\newpage
\section{Лабораторная работа №1.\\Сравнение циклов и рекурсии}
\subsection{Цели}
\begin{itemize}
	\item Оценить недостатки процедурного программирования
	\item Научиться строить рекурсивные алгоритмы
\end{itemize}

\subsection{Порядок выполнения}
\begin{enumerate}
	\item Написать программу по заданию с использованием цикла
	\item Провести трассировку программы
	\item Составить рекурсивную функцию для решения выданного задания
	\item Реализовать составленную рекурсивную функцию на языке программирования
	\item Написать отчет
\end{enumerate}

\subsection{Рекомендации по выполнению}
\begin{itemize}
	\item Массивы фиксированной длины
	\item Трассировка отключается макросом
	\item Данные задаются внутри исходного кода
\end{itemize}

\subsection{Состав отчета}
\begin{itemize}
	\item Титульный лист (фамилия, группа, номер варианта, наименование работы, задание)
	\item Текст рекурсивной функции
	\item Текст итеративной функции
	\item Результаты выполнения
\end{itemize}

\subsection{Варианты заданий}
\begin{enumerate}
	\item Напишите программу печатающую $n$-ое число Фибоначчи.
	\item Напишите программу вычисляющую факториал натурального числа.
	\item Напишите программу перемножающую два целых неотрицательных числа без использования операции умножения.
	\item Напишите программу, печатающую значение многочлена степени $n\geq0$ в заданной точке $x_{0}$. Коэффициенты многочлена хранятся в массиве $a$ в порядке убывания степений и являются целыми числами, так же как и значение $x_{0}$.
	\item Напишите программу печатающую значение производной многочлена степени $n\geq0$ в заданной точке $x_{0}$. Коэффициенты многочлена хранятся в массиве $a$ в порядке убывания степений и являются целыми числами, так же как и значение $x_{0}$.
	\item Напишите программу возводящую целое число в целую неотрицательную степень.
	\item Напишите программу принимающую на вход натуральное число и выводающую Yes если число является простым, и No - если не является.
	\item Напишите программу генерации всех правильных скобочных структур длины $2n$. Например для $n=3$ таких структур может быть 5: ()()(), (())(), ()(()), ((())), (()()).
	\item Имеется три стержня А, В, С. На стержень А нанизано $n$ дисков радиуса $1, 2,..., n$ таким образом, что диск радиуса $i$ является $i$-м сверху. Требуется переместить все диски на стержень В, сохраняя их порядок расположения (диск с большим радиусом находится ниже). За один раз можно перемещать только один диск с любого стержня на любой другой стержень. При этом должно выполняться следующее условие: на каждом стержне ни в какой момент времени никакой диск не может находиться выше диска с меньшим радиусом. 
	\item Напишите программу выводящую сумму квадратов всех натуральных чисел от 1 до введённого $n$.
	\item Напишите программу печатающую $n$-ое простое число.
	\item Напишите программу, печатающую старшую цифру в десятичной записи введенного натурального числа.
	\item Напишите программу, печатающую количество цифр в десятичной записи введенного натурального числа.
	\item Напишите программу, печатающую количество натуральных решений неравенства $x^{2}+y^{2}<n$ для введенного $n$.
	\item Напишите программу, вводящую натуральное число , и печатающую количество точек с целочисленными координатами внутри замкнутого шара радиуса с центром в начале координат.
	\item Напишите программу, печатающую квадраты всех целых чисел от нуля до введенного натурального $n$, не использующую операций умножения.
	\item Напишите программу, находящую количество счастливых билетов с шестизначными номерами. Билет называется счастливым, если сумма его первых трех цифр равна сумме трех последних.
\end{enumerate}

\subsection{Пример}
\paragraph{Задание:}
Напишите программу проверяющую является ли введённое число факториалом какого либо числа.
\subsubsection{Итеративное решение}
Возьмём математическое определение факториала:
\begin{equation}
n! = 1\cdot 2\cdot\ldots\cdot n =\prod_{i=1}^n i
\end{equation}

Получается что факториал числа $n$ должен делиться нацело на все натуральные числа до $n$ включая $n$. Напишем программу рализующую такую проверку:

\begin{lstlisting}[language=c,{caption=итеративная программа}]
#include <stdio.h>

int check_factorial_iterate(int number){
	if(number < 0) 
		return 0;
	if(number == 0) 
		return 1;
	int i = 1;
	int n = 1;
	for(;n<number;n*=i,i++){
		
		if(number%i != 0)
			return 0;
	}
	return 1;
}

int main(){
	if(check_factorial_iterate(362880))
		printf("%s","yes");
	else
		printf("%s","no");
	return 0;
}
\end{lstlisting}

\subsubsection{Рекурсивное решение}

Возьмём рекурсивное определение факториала:

\begin{equation}
n!= \begin{cases}
1 & n = 0,\\
n \cdot (n-1)! & n > 0.
\end{cases}
\end{equation}

\begin{lstlisting}[language=c,{caption=рекурсивная программа}]
#include <stdio.h>

int check_factorial_recursive(int number, int i){
	if(number < 0) 
		return 0;	
	if(number == 0)
		return 1;

	if(number == 1)
		return 1;
	
	if( number%i != 0)
		return 0;
	else		
		return check_factorial_recursive(number/i,i+1);
	
}

int main(){
	int number = 362881;

	if(check_factorial_recursive(number,1))
		printf("%s\n","yes");
	else
		printf("%s\n","no");

	return 0;
}
\end{lstlisting}

\newpage
\section{Лабораторная работа №2\\S-выражения в LISP}
\subsection{Цели}

\begin{itemize}
	\item Освоить S-выражения
	\item Научиться основам работы в clisp
	\item Познакомиться с функциями обработки списков
\end{itemize}

\subsection{Задание}
\begin{enumerate}
	\item Составить список по заданию в синтаксисе lisp
	\item Написать функции для получения каждого из элементов списка №1
	\item Написать функцию для получения списка №2
\end{enumerate}

\subsection{Пример}
\paragraph{Задание}
\begin{enumerate}
	\item Список №1 - \{\{1,\{2,3,4,5,6,7,8\}\},9\}
	\item Список №2 - \{2,8,3,\{4,1\},6\}
\end{enumerate}

\begin{lstlisting}[language=lisp,{caption=Задание 1}]
'((1 (2 3 4 5 6 7 8)) 9)
\end{lstlisting}

\begin{lstlisting}[language=lisp,{caption=Задание 2}]
(car '((1 (2 3 4 5 6 7 8)) 9))
;; (1 (2 3 4 5 6 7 8))
(car (car '((1 (2 3 4 5 6 7 8)) 9)))
;; 1
(car (cdr (car '((1 (2 3 4 5 6 7 8)) 9))))
;; (2 3 4 5 6 7 8)
(car (car (cdr (car '((1 (2 3 4 5 6 7 8)) 9)))))
;; 2
(setq a '((1 (2 3 4 5 6 7 8)) 9))
;; ((1 (2 3 4 5 6 7 8)) 9)
(car (car (cdr (car a))))
;; 2
(car (cdr (car (cdr (car a)))))
;; 3
(car (cdr (cdr (car (cdr (car a))))))
;; ...
\end{lstlisting}

\begin{lstlisting}[language=lisp,{caption=Задание 3}]
(cons 2	(cons 8 (cons 3 (cons (cons 4 (cons 1 nil)) (cons 6 nil)))))
;;(2 8 3 (4 1) 6)
\end{lstlisting}

\subsection{Состав отчета}
\begin{itemize}
	\item Титульный лист (фамилия, группа, номер варианта, наименование работы, задание)
	\item Текст рекурсивной функции
	\item Результаты выполнения
\end{itemize}

\subsection{Варианты заданий}
\begin{enumerate}
	\item \begin{itemize}
		\item Список №1 - \{1,\{2,3,4\},5,\{6,\{7,8\},9\}\}
		\item Список №2 - \{2,8,\{3,4,1\},6\}
	\end{itemize}

	\item \begin{itemize}
		\item Список №1 - \{\{1,2,\{3,4,5,6\},7\},8,9\}
		\item Список №2 - \{2,\{8,3,4,1\},6\}
	\end{itemize}

	\item \begin{itemize}
		\item Список №1 - \{\{\{1,2,3,4\},5,6,7,8\},9\}
		\item Список №2 - \{\{7,8\},\{3,4\},\{1,6\}\}
	\end{itemize}

	\item \begin{itemize}
		\item Список №1 - \{1,2,\{3,4,5\},\{\{6,7,8,9\}\}\}
		\item Список №2 - \{\{2,8\},6,7,8,9\}
	\end{itemize}

	\item \begin{itemize}
		\item Список №1 - \{1,2,\{3,4,5\},\{6,7,\{8,9\}\}\}
		\item Список №2 - \{2,8,\{3,4,5\},6\}
	\end{itemize}

	\item \begin{itemize}
		\item Список №1 - \{1,\{\{2,3,4\},5,6,7\},8,9\} 
		\item Список №2 - \{2,\{\{8\},3,4,1,6\}\}
	\end{itemize}

	\item \begin{itemize}
		\item Список №1 - \{1,\{2,3,\{4,5,6\},7,8\},9\}
		\item Список №2 - \{2,\{8,3,4,\{1\}\},6\}
	\end{itemize}

	\item \begin{itemize}
		\item Список №1 - \{\{\{1,2,3,4,\}5,6,7,8\},9\}
		\item Список №2 - \{2,8,3,\{4,1\},6\}
	\end{itemize}

	\item \begin{itemize}
		\item Список №1 - \{\{\{1,2,3,4,5,6,7,8\},9\}\}
		\item Список №2 - \{2,8,3,4,\{1,6\}\}
	\end{itemize}

	\item \begin{itemize}
		\item Список №1 - \{1,2,3,\{4,\{5,6,\{7,8,9\}\}\}\}
		\item Список №2 - \{2,\{8,3\},4,1,6\}
	\end{itemize}

	\item \begin{itemize}
		\item Список №1 - \{1,\{2,3,\{4,5,6,\{7,8\},9\}\}\}
		\item Список №2 - \{2,8,3,4,\{\{1\},6\}\}
	\end{itemize}

	\item \begin{itemize}
		\item Список №1 - \{1,\{\{2,3\},4,5,6\},7,8,9\}
		\item Список №2 - \{\{2\},8,\{3,\{4,1\}\},6\}
	\end{itemize}

	\item \begin{itemize}
		\item Список №1 - \{1,\{2,\{3\},4,5\},6,\{7,8\},9\}
		\item Список №2 - \{2,\{\{8,3\},\{4,1\}\},6\}
	\end{itemize}

	\item \begin{itemize}
		\item Список №1 - \{\{1,\{2\},3,4\},5,6,\{7,\{8,9\}\}\}
		\item Список №2 - \{2,\{8,3,\{4\},1\},\{6\}\}
	\end{itemize}

	\item \begin{itemize}
		\item Список №1 - \{1,\{2,3,\{4,5,\{6,\{7\}\}\},8\},9\}
		\item Список №2 - \{2,\{8,\{3,\{4\}\},1\},6\}
	\end{itemize}

	\item \begin{itemize}
		\item Список №1 - \{\{1\},2,\{3,4\},\{5,6,\{7,8\},9\}\}
		\item Список №2 - \{\{2,8\},\{3,\{4,1\},6\}\}
	\end{itemize}

	\item \begin{itemize}
		\item Список №1 - \{1,\{2,3\},4,\{5\},6,7,8,9\}
		\item Список №2 - \{2,\{8\},\{3,4,\{1\}\},6\}
	\end{itemize}

	\item \begin{itemize}
		\item Список №1 - \{1,2,\{3\},\{4\},\{5\},6,\{7,\{8\}\},9\}
		\item Список №2 - \{\{2,\{8,3\},4,1\},\{6\}\}
	\end{itemize}

	\item \begin{itemize}
		\item Список №1 - \{1,2,\{3,\{4\},5\},\{6,\{7\},8\},9\}
		\item Список №2 - \{\{2,\{8,3,4,\{1\}\}\},6\}
	\end{itemize}

	\item \begin{itemize}
		\item Список №1 - \{\{1,2\},\{\{3,\{4\},5\},\{6,\{7\},8\}\},\{9\}\}
		\item Список №2 - \{2,\{\{8\},3,\{4,1\}\},\{6\}\}
	\end{itemize}

\end{enumerate}

\newpage
\section{Лабораторная работа №3\\Функции в LISP}

\subsection{Цели}
\begin{itemize}
	\item Познакомиться именованными фунциями
	\item Познакомиться с анонимными функциями
\end{itemize}

\subsection{Задание}
\begin{enumerate}
	\item Написать функцию по первому заданию
	\item Написать функцию принимающую в качестве аргумента список заданного вида и возвращающую список такого же вида, но с изменёнными значениями.
	\item Написать анонимную функцию, которая передаётся парметром в функцию из второго задания, и выполняет некоторые действия над элементами списка.
\end{enumerate}

\subsection{Пример}
\paragraph{Задание}
\begin{enumerate}
	\item Функция принимает два аргумента и возвращает их сумму.
	\item Функция принимает список вида (x x x x x ... ) и увличивает каждый элемент списка на 1.
	\item Написать анонимную функцию которая преобразует каждый элемент в список.\\ ((x) (x) (x)...)
\end{enumerate}

\begin{lstlisting}[language=lisp,{caption=Задание 1}]
(defun summa (a b) 
	(+ a b)
)

;; пример 
(summa 4 5)
\end{lstlisting}

\begin{lstlisting}[language=lisp,{caption=Задание 2}]
(defun mapx (x) 
	(if x
		(cons (+ (car x) 1) (mapx (cdr x)))
		nil
	)
)
;; пример вызова
(mapx '(1 2 3 4))
\end{lstlisting}

\begin{lstlisting}[language=lisp,{caption=Задание 3}]
(defun mapx (x f) 
	(if x
		(cons (funcall f (car x)) (mapx (cdr x) f))
		nil
	)
)
;; пример вызова
(mapx '(1 2 3 4) (lambda (x)(+ x 1)))
\end{lstlisting}

\subsection{Состав отчета}
\begin{itemize}
	\item Титульный лист (фамилия, группа, номер варианта, наименование работы, задание)
	\item Текст рекурсивной функции
	\item Результаты выполнения
\end{itemize}

\subsection{Варианты заданий}
\begin{enumerate}
	\item \begin{itemize}
		\item Функция принимает два числа, и если их сумма чётна, то возвращает их разницу, иначе - сумму.
		\item Дан список ((x x) (x x) (x x) ... ). Увеличить каждый элемент на единицу.
		\item Объеденить каждый подсписок суммированием: ((1 2) (1 3) (3 4)) -> ((3) (4) (7))
	\end{itemize}

	\item \begin{itemize}
		\item Функция принимает два числа и возвращает наибольшее из них
		\item Дан список ((x x x x ...) (x x x x ...)). Увеличить каждый элемент на единицу
		\item Умножить каждый элемент на произвольное число
	\end{itemize}

	\item \begin{itemize}
		\item Функция принимает 3 числа и возвращает список с этими числами
		\item Дан список (x (x (x (x...)))). Увеличить каждый элемент на единицу
		\item Умножить каждый элемент на 2
	\end{itemize}

	\item \begin{itemize}
		\item Функция принимает 1 число и возвращает квадрат этого числа если оно чётное, и куб, если нечётное
		\item Дан список ((((... x) x) x) x). Увеличить каждый элемент на единицу
		\item Поделить каждый элемент на 2
	\end{itemize}

	\item \begin{itemize}
		\item Функция принимает 2 числа и возвращает их произведение, если их сумма чётна, и квадрат первого, если сумма нечётна
		\item Дан список ((x (x (x))) (x (x (x))) (x (x (x))) ... ). Увеличить каждый элемент на единицу
		\item Увеличить каждый элемент в 2 раза
	\end{itemize}

	\item \begin{itemize}
		\item Функция принимает список и число, и добавляет число к списку.
		\item Дан список ((x x x) (x x x) ... ). Увеличить каждый элемент на единицу.
		\item Объеденить каждый подсписок суммированием:\\ ((1 2 1) (1 3 2) (3 4 1)) -> ((4) (6) (8))
	\end{itemize}

	\item \begin{itemize}
		\item Функция принимает список 3 числа и возвращает список вида (x (x (x)))
		\item Дан список ((x) (x) ... ). Увеличить каждый элемент на единицу.
		\item Преобразовать список в простой список элементов: ((1) (3) (4)) -> (1 3 4)
	\end{itemize}

	\item \begin{itemize}
		\item Функция принимает список 3 числа и возвращает список вида (((x) x) x)
		\item Дан список ((x (x x)) (x (x x)) ... ). Увеличить каждый элемент на единицу.
		\item Увеличить каждый элемент в произвольное число раз
	\end{itemize}

	\item \begin{itemize}
		\item Функция принимает список 3 числа и возвращает список вида ((x) x (x))
		\item Дан список (((x) x (x)) ((x) x (x)) ... ). Увеличить каждый элемент на единицу.
		\item Увеличить каждый элемент в произвольное число раз
	\end{itemize}

	\item \begin{itemize}
		\item Функция принимает список 3 числа и возвращает список вида (x (x) x)
		\item Дан список (((x) (x)) ((x) (x)) ... ). Увеличить каждый элемент на единицу.
		\item Увеличить каждый элемент в произвольное число раз
	\end{itemize}

	\item \begin{itemize}
		\item Функция принимает список 3 числа и возвращает список вида ((x) x (x))
		\item Дан список ((x) (x) ... ). Увеличить каждый элемент на единицу.
		\item Увеличить каждый элемент в произвольное число раз
	\end{itemize}

	\item \begin{itemize}
		\item Функция принимает два числа и возвращает наименьшее из них
		\item Дан список ((((... x) x) x) x). Увеличить каждый элемент на 2.
		\item Увеличить каждый элемент в произвольное число раз
	\end{itemize}

	\item \begin{itemize}
		\item Функция принимает список и число, и добавляет число к списку с конца.
		\item Дан список ((x) (x) ... ). Увеличить каждый элемент на единицу.
		\item Поделить каждый элемент на 2
	\end{itemize}

	\item \begin{itemize}
		\item Функция принимает 3 числа и возвращает список с этими числами
		\item Дан список ((x) (x) ... ). Увеличить каждый элемент на единицу.
		\item Умножить каждый чётный элемент на произвольное число
	\end{itemize}

	\item \begin{itemize}
		\item Функция принимает два числа и возвращает первое, если оно кратно второму, и второе, если не кратно.
		\item Дан список ((x x x x ...) (x x x x ...)). Увеличить каждый элемент на единицу
		\item Объеденить каждый подсписок суммированием:\\((1 2 1...) (1 3 2 ...) (3 4 1 ...)) -> ((4) (6) (8))
	\end{itemize}

\end{enumerate}

\newpage
\section{Лабораторная работа №4 \\ 
	$\lambda$-выражения и $\beta$-редукция в $\lambda$-исчислении в языке LISP}
\subsection{Цели}
\begin{itemize}
	\item Изучить $\lambda$-исчисление
	\item Изучить $\beta$-редукцию в $\lambda$-исчислении
\end{itemize}

\section{Задание}
\begin{enumerate}
	\item Выполнить $\beta$-редукции несколькими способами
	\item Составить программу на LISP для вычисления функции
\end{enumerate}

\subsection{Пример}
\paragraph{Задание}
\[((((\lambda xyz.x(yz))(\lambda x.x\cdot 1))(\lambda x.x+x))2)\]

\begin{lstlisting}[language=lisp,{caption=Программа на LISP}]
( 
  (lambda (x y z) (funcall x (funcall y z))) 
  (lambda (x) (* x 1)) 
  (lambda (x) (+ x x)) 
2)

;; результат
4
\end{lstlisting}

\begin{align*}
(((\lambda xyz.x(yz))(\lambda x.x))(\lambda x.x+x))2=\\
=((\lambda yz.(\lambda x.x)(yz))(\lambda x.x+x))2 =\\
=(\lambda z.(\lambda x.x)((\lambda x.x+x)z))2
\end{align*}

\[(\lambda x.x)((\lambda x.x+x)2)=(\lambda x.x+x)2=2+2=4\]
\[(\lambda z.((\lambda x.x+x)z))2=(\lambda z.z+z)2=2+2=4\]


\subsection{Состав отчета}
\begin{itemize}
	\item Титульный лист (фамилия, группа, номер варианта, наименование работы, задание)
	\item Текст LISP программы
	\item $\beta$-редукции
\end{itemize}

\paragraph{Варианты заданий}
\begin{enumerate}
	\item \[ ((((\lambda xyz.x(yz))(\lambda x.x+x))(\lambda x.x\cdot x))3) \]
	\item \[ ((((\lambda xyz.x(yz))(\lambda x.x\cdot x))(\lambda x.x+x))4) \]
	\item \[ ((((\lambda xyz.x(yz))(\lambda x.x\cdot x))(\lambda x.x\cdot x))5) \]
	\item \[ ((((\lambda xyz.x(yz))(\lambda x.x+8))(\lambda x.9\cdot x))6) \]
	\item \[ ((((\lambda xyz.x(yz))(\lambda x.x))(\lambda x.x\cdot x))7)\]
	\item \[ ((((\lambda xyz.x(yz))(\lambda x.x+x))(\lambda x.x+x))8)\]
	\item \[ (((((\lambda xyz.xzy))(\lambda xy.x\div y))3)9)\]
	\item \[ (((((\lambda xyz.xzy))(\lambda xy.x\div y))((\lambda x.x)3))9)\]
	\item \[ (((((\lambda xyz.xzy))(\lambda xy.x\div y))3)((\lambda x.x)9))\]
	\item \[ ((((\lambda xyz.xz(yz)))(\lambda xy.x+y))(\lambda x.x+2)3)\]
	\item \[ ((((\lambda xyz.xz(yz)))(\lambda xy.x))(\lambda x.x)4))\]
	\item \[ ((((\lambda xyz.xz(yz)))(\lambda xy.y))(\lambda x.x+2)5)\]
	\item \[ ((((\lambda xyz.xz(yz)))(\lambda xy.x\cdot y))(\lambda x.x+2)6)\]
	\item \[ ((((\lambda xyz.xz(yz)))(\lambda xy.x+y))(\lambda x.x\cdot 2)7)\]
	\item \[ ((((\lambda xyz.xz(yz)))(\lambda xy.x-y))(\lambda x.x-2)8)\]
\end{enumerate}


\newpage
\section{Лабораторная работа №5 \\
	Основные возможности Haskell}
\subsection{Цели}
\begin{itemize}
	\item Приобрести навыки работы с интерпретатором языка Haskell. Получить представление об основных типах языка Haskell. Научиться определять простейшие функции.
\end{itemize}

\subsection{Задание}
\begin{enumerate}
	\item Изучить теоретические сведения
	\item Выполнить задания в соответствии с вариантом
\end{enumerate}

\subsection{Теоретические сведения}
\subsubsection{Типы}
\begin{itemize}
	\item Типы Integer и Int используется для представления целых чисел, причем значения типа Integer не ограничены по длине.
	\item Типы Float и Double используется для представления вещественных чисел.
	\item Тип Bool содержит два значения: True и False, и предназначен для представления результата логических выражений.
	\item  Тип Char используется для представления символов.
\end{itemize}

\subsubsection{Списки}
Чтобы задать список в Haskell, необходимо в квадратных скобках перечислить его элементы через запятую. Все эти элементы должны принадлежать одному и тому же типу. Тип списка с элементами, принадлежащими типу a, обозначается как [a]. Примеры: [1,2]; ['1','2','3'].

\paragraph{Операции со списками}
Оператор : (двоеточие) используется для добавления элемента в начало списка. Его левым аргументом должен быть элемент, а правым - список:

\begin{lstlisting}[language=haskell,{caption=Пример}]
> 1:[2,3]
[1,2,3] :: [Integer]
> '5':['1','2','3','4','5']
['5','1','2','3','4','5'] :: [Char]
> False:[]
[False] :: [Bool]
\end{lstlisting}

С помощью оператора (:) и пустого списка можно построить любой список:

\begin{lstlisting}[language=haskell,{caption=Пример}]
> 1:(2:(3:[]))
[1,2,3] :: Integer
>1:2:3:[]
[1,2,3] :: Integer
\end{lstlisting}

Элементами списка могут быть любые значения — числа, символы, кортежи, другие списки и т.д.

\begin{lstlisting}[language=haskell,{caption=Пример}]
> [(1,'a'),(2,'b')]
[(1,'a'),(2,'b')] :: [(Integer,Char)]
> [[1,2],[3,4,5]]
[[1,2],[3,4,5]] :: [[Integer]]
\end{lstlisting}

Для работы со списками в языке Haskell существует большое количество функций. В данной лабораторной работе рассмотрим только некоторые из них.

\begin{itemize}
	\item Функция head возвращает первый элемент списка.
	\item Функция tail возвращает список без первого элемента.
	\item Функция length возвращает длину списка.
\end{itemize}

Функции head и tail определены для непустых списков. При попытке применить их к пустому списку интерпретатор сообщает об ошибке. Примеры работы с указанными функциями:

\begin{lstlisting}[language=haskell,{caption=Пример}]
> head [1,2,3]
1 :: Integer
> tail [1,2,3]
[2,3] :: [Integer]
> tail [1]
[] :: Integer
> length [1,2,3]
3 :: Int
\end{lstlisting}

Для соединения (конкатенации) списков в Haskell определен оператор ++.

\begin{lstlisting}[language=haskell,{caption=Пример}]
> [1,2]++[3,4]
[1,2,3,4] :: Integer
\end{lstlisting}

\subsubsection{Функции} Рассмотрим пример:

\begin{lstlisting}[language=haskell,{caption=Пример}]
square :: Integer -> Integer
square x = x * x
\end{lstlisting}

Первая строка (square :: Integer -> Integer) объявляет, что мы определяем функцию square, принимающую параметр типа Integer и возвращающую результат типа Integer. Вторая строка (square x = x * x) является непосредственно определением функции. Функция square принимает один аргумент и возвращает его квадрат.

В общем виде тип функции, принимающей n аргументов, принадлежащих типам t1, t2, . . . , tn, и возвращающей результат типа a, записывается в виде t1->t2->...->tn->a.

\begin{lstlisting}[language=haskell,{caption=Пример}]
add :: Integer -> Integer -> Integer
add x y = x + y
\end{lstlisting}

\subparagraph{Условное выражение} В общем виде выглядит так: \\
if условие then выражение else выражение.

Функцию signum, вычисляющую знак переданного ей аргумента:

\begin{lstlisting}[language=haskell,{caption=Пример}]
signum :: Integer -> Integer
signum x = if x > 0
        then 1
        else if x < 0
                then -1
                else 0
\end{lstlisting}

Следует обратить внимание на отступы. Именно по отступам компилятор определяет к какому 
if'у отностится тот или иной else.

Условие в определении условного оператора представляет собой любое выражение типа Bool. Примером таких выражений могут служить сравнения. При сравнении можно использовать следующие операторы:

\begin{itemize}
	\item <, >, <=, >= - эти операторы имеют такой же смысл, как и в языке Си (меньше, больше, меньше или равно, больше или равно);
	\item == - оператор проверки на равенство;
	\item /= - оператор проверки на неравенство.
\end{itemize}

\subsection{Пример}
\paragraph{Задание}
Написать функцию на языке Haskell возвращающую знак переданного целого числа.

\begin{lstlisting}[language=haskell,{caption=Пример}]
signum :: Integer -> Integer
signum x = if x > 0
        then 1
        else if x < 0
                then -1
                else 0

main = print $ Prelude.signum 1
\end{lstlisting}

\subsection{Состав отчета}
\begin{itemize}
	\item Титульный лист (фамилия, группа, номер варианта, наименование работы, задание)
	\item Текст программы на языке Haskell
	\item Результат работы программы на языке Haskell
\end{itemize}

\subsection{Варианты заданий}
\begin{enumerate}
	\item Функция max3, по трем целым возвращающая наибольшее из них.
	\item Функция min3, по трем целым возвращающая наименьшее из них.
	\item Функция sort2, по двум целым возвращающая пару, в которой наименьшее из них стоит на первом месте, а наибольшее - на втором.
	\item Функция bothTrue :: Bool -> Bool -> Bool, которая возвращает True тогда и только тогда, когда оба ее аргумента будут равны True. Не используйте при определении функции стандартные логический операции  (\&\& , || и т.п.).
	\item Функция     solve2::Double->Double->(Bool,Double), которая по двум числам, представляющим собой коэффициенты линейного уравнения ax + b = 0, возвращает пару, первый    элемент которой равен True, если решение существует и False в противном случае; при этом второй элемент равен либо значению корня, либо 0.0.
	\item Функция isParallel, возвращающая True, если два отрезка, концы которых задаются в аргументах функции, параллельны (или лежат на одной прямой). Например, значение выражения isParallel (1,1) (2,2) (2,0) (4,2) должно быть равно True, поскольку отрезки (1, 1) - (2, 2) и (2, 0) - (4, 2) параллельны.
	\item Функция isIncluded, аргументами которой служат параметры двух окружностей на плоскости (координаты центров и радиусы). Функция возвращает True, если вторая окружность целиком содержится внутри первой.
 	\item Функция isRectangular, принимающая в качестве параметров координаты трех точек на плоскости, и возвращающая True, если образуемый ими треугольник - прямоугольный.
	\item Функция isTriangle, определяющая, можно ли их отрезков с заданными длинами x, y и z построить треугольник.
	\item Функция isSorted, принимающая на вход три числа и возвращающая True, если они упорядочены по возрастанию или по убыванию.
	\item Функция принимает два числа, и если их сумма чётна, то возвращает их разницу, иначе - сумму.
	\item Функция принимает список и число, и добавляет число к списку.
	\item Функция принимает список и число, и добавляет число к списку с конца.
	\item Функция принимает 3 числа и возвращает список с этими числами
	\item Функиця принимает 2 числа, и возвращает наиболшее кратное двойке.
\end{enumerate}

\newpage
\subsection{Лабораторная работа №6 \\ Использование комбинаторов в языке Haskell}
\subsection{Цели}
\begin{itemize}
	\item Познакомиться с комбинаторами на языке Haskell
\end{itemize}

\subsection{Задание}
\begin{enumerate}
	\item Построить комбинаторы $i$, $b$, $k$, $c$, $w$, $s$;
	\item функции $p\ x = x + 1$, $u\ x\ y = x + y$ (функции $p$ и $u$ могут быть использованы не более одного раза);
	\item проверить их работоспособность;
	\item построить выражение в соответствии с заданием.
\end{enumerate}

\subsection{Пример}
\paragraph{Задание}

Построить выражение 
$v\ 6\ 7 = 7$

\begin{lstlisting}[language=haskell,{caption=Пример}]
i x = x
k x y = x
s x y z = x z (y z)
b x y z = x (y z)
c x y z = x z y
w x y = x y y

p x = x + 1
u x y = x + y

main = print $ k i 6 7
\end{lstlisting}

\subsection{Состав отчета}
\begin{itemize}
	\item Титульный лист (фамилия, группа, номер варианта, наименование работы, задание)
	\item Текст программы на языке Haskell
	\item Результат работы программы на языке Haskell
\end{itemize}

\subsection{Варианты заданий}
\begin{enumerate}
	\item $v\ 6\ 7 = 7$
	\item $v\ 6 = 8$
	\item $v\ 5 = 10$
	\item $v\ 4 = 9$
	\item $v\ 3\ 5 = 9$
	\item $v\ 4\ 7 = 5$
	\item $v\ 3\ 4\ 5 = 9$
	\item $v\ 7\ 2\ 6 = 2$
	\item $v\ 8\ 2 = 3$
	\item $v\ 8\ 7 = 14$
	\item $v\ 7\ 9 = 14$
	\item $v\ 2 = 4$
	\item $v\ 9 = 18$
	\item $v\ 5 = 11$
	\item $v\ 2\ 6 = 9$
	\item $v\ 3\ 5 = 4$
	\item $v\ 8\ 2\ 5 = 7$
	\item $v\ 7\ 3\ 6 = 3$
	\item $v\ 8\ 1 = 2$
	\item $v\ 8\ 6 = 12$
	\item $v\ 4\ 9 = 8$
\end{enumerate}

\newpage
\section{Лабораторная работа №7 \\ 
	Формализация предметной области для языка Пролог}
\subsection{Цели}
\begin{itemize}
	\item Познакомиться языком Пролог
\end{itemize}

\subsection{Задание}
\begin{enumerate}
	\item Для соответствующей варианту предметной области составить 3-5 аксиом.
	\item Составить несколько вопросов.
	\item Записать на языке Пролог.
\end{enumerate}

\subsection{Пример}
\paragraph{Задание}
\begin{itemize}
	\item Предметная область: раковина;
	\item описание предметной области: Раковина засоряется тем, что пропустит фильтр. Фильтр пропускает очистки морковки при использовании любых инструментов для чистки и очистки любого овощя очищенного картофелечисткой. Морковь почистили ножом, картофель - картофелечисткой;
	\item вопросы: Чем засорится раковина? Засорится ли раковина?.
\end{itemize}

\paragraph{Решение} Объявим предикаты
\begin{itemize}
	\item $C(x,y)$ - чистить икс игреком
	\item $F(x)$ - фильтр пропустит х
	\item $R(x)$ - раковина засорилась х
\end{itemize}

\begin{align*}
A1&: F(x) \to R(x)\\
A2&: C(x, kartochistka) \to F(x)\\
A3&: C(morkva, y) \to F(morkva)\\
A4&: C(morkva, nozh)\\
A5&: C(kartofan, kartochistka)\\
B1&: R(x)\\
B2&: R(kartofan)\\
\end{align*}

Напишем программу на языке Пролог:

\begin{lstlisting}[language=prolog,{caption=Пример}]
r(X) :- f(X).
f(X) :- c(X, 'kartochistka').
f('morkva') :- c('morkva', _).
c('morkva', 'nozh').
c('kartofan', 'kartochistka').
\end{lstlisting}

Зададим пару вопросов программе:

\begin{lstlisting}[language=prolog,{caption=Результат}]
$ gprolog
GNU Prolog 1.2.18
By Daniel Diaz
Copyright (C) 1999-2004 Daniel Diaz
| ?- consult('rack.pl').
compiling /home/kdb/prolog/rack.pl for byte code...
/home/kdb/prolog/rack.pl compiled, 5 lines read - 739 bytes written, 25 ms

yes
| ?- r('morkva').

yes
| ?- r(X).

X = kartofan ? a

X = morkva

yes
| ?- trace.
The debugger will first creep -- showing everything (trace)

yes

{trace}
| ?- r('kartofan').
      1    1  Call: r(kartofan) ? 
      2    2  Call: f(kartofan) ? 
      3    3  Call: c(kartofan,kartochistka) ? 
      3    3  Exit: c(kartofan,kartochistka) ? 
      2    2  Exit: f(kartofan) ? 
      1    1  Exit: r(kartofan) ? 

yes
{trace}
| ?- r(X).         
      1    1  Call: r(_16) ? 
      2    2  Call: f(_16) ? 
      3    3  Call: c(_16,kartochistka) ? 
      3    3  Exit: c(kartofan,kartochistka) ? 
      2    2  Exit: f(kartofan) ? 
      1    1  Exit: r(kartofan) ? 

X = kartofan ?  a
      1    1  Redo: r(kartofan) ? 
      2    2  Redo: f(kartofan) ? 
      3    3  Call: c(morkva,_69) ? 
      3    3  Exit: c(morkva,nozh) ? 
      2    2  Exit: f(morkva) ? 
      1    1  Exit: r(morkva) ? 

X = morkva

yes
{trace}
| ?- 
\end{lstlisting}

\subsection{Состав отчета}
\begin{itemize}
	\item Титульный лист (фамилия, группа, номер варианта, наименование работы, задание)
	\item Задание
	\item Аксиомы
	\item Вопросы
	\item Программа на языке Пролог
\end{itemize}

\subsection{Варианты заданий}
\begin{enumerate}
	\item книжный шкаф
	\item зоопарк
	\item театр
	\item цирк
	\item телевизор
	\item баня
	\item таблица умножения
	\item сдача экзамена
	\item ноутбук
	\item файловая система
	\item правительства
	\item система прав пользователей
	\item продуктовый магазин
	\item учебная аудитория
	\item родственники
	\item структура каталогов
	\item арифметические действия
	\item текст
	\item гипертекст
	\item холодильник
	\item лес
	\item огород
\end{enumerate}

\end{document}
