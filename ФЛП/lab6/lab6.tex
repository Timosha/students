\documentclass[a4paper,12pt]{article}

\usepackage[T2A]{fontenc}
\usepackage[utf8x]{inputenc}
\usepackage[russian,english]{babel}
\usepackage{graphicx}
\usepackage{color}
\usepackage{xcolor}
\usepackage{listings}
\usepackage{fancyhdr}
\usepackage{amsmath}
\usepackage{textcomp}
\usepackage{amssymb}
\usepackage{indentfirst} % включить отступ у первого абзаца

%%\usepackage[
%%		a4paper, includefoot,
%%		left=3cm, right=1cm, top=2cm, bottom=1.5cm,
%%		headsep=1cm, footskip=1cm
%%	]{geometry}

\title{Лабораторная работа №6}
\author{Вагин Д.А.}
\date{04/2011}
\pagestyle{empty}
\pagestyle{fancy}
\lhead{Лабораторная работа №6} %верхний колонтитул слева
\rhead{\title}


%%\lstset{language=c++,inputencoding=utf8x, extendedchars=\true,captionpos=b,tabsize=3,frame=lines,keywordstyle=\color{blue},commentstyle=\color{green},stringstyle=\color{red},numbers=left,numberstyle=\t	iny,numbersep=5pt,breaklines=true,showstringspaces=false,basicstyle=\footnotesize,emph={label}}

\lstloadlanguages{haskell}
\lstset{
	language=haskell,inputencoding=utf8x,
	extendedchars=\true,captionpos=b,tabsize=4,
	frame=lines,
	keywordstyle=\color{blue},commentstyle=\color{green},stringstyle=\color{red},
	breaklines=true,showstringspaces=false,basicstyle=\footnotesize
}

\begin{document}

%%\maketitle

\paragraph{Использование комбинаторов в языке Haskell}
\subparagraph{Цели}
\begin{itemize}
	\item Познакомиться с комбинаторами на языке Haskell
\end{itemize}

\paragraph{Задание}
\begin{enumerate}
	\item Построить комбинаторы $i$, $b$, $k$, $c$, $w$, $s$;
	\item функции $p\ x = x + 1$, $u\ x\ y = x + y$ (функции $p$ и $u$ могут быть использованы не более одного раза);
	\item проверить их работоспособность;
	\item построить выражение в соответствии с заданием.
\end{enumerate}

\paragraph{Пример}
\subparagraph{Задание}
Построить выражение 
$v\ 6\ 7 = 7$

\begin{lstlisting}[language=haskell,{caption=Пример}]
i x = x
k x y = x
s x y z = x z (y z)
b x y z = x (y z)
c x y z = x z y
w x y = x y y

p x = x + 1
u x y = x + y

main = print $ k i 6 7
\end{lstlisting}

\subparagraph{Состав отчета}
\begin{itemize}
	\item Титульный лист (фамилия, группа, номер варианта, наименование работы, задание)
	\item Текст программы на языке Haskell
	\item Результат работы программы на языке Haskell
\end{itemize}

\paragraph{Варианты заданий}
\begin{enumerate}
	\item $v\ 6\ 7 = 7$
	\item $v\ 6 = 8$
	\item $v\ 5 = 10$
	\item $v\ 4 = 9$
	\item $v\ 3\ 5 = 9$
	\item $v\ 4\ 7 = 5$
	\item $v\ 3\ 4\ 5 = 9$
	\item $v\ 7\ 2\ 6 = 2$
	\item $v\ 8\ 2 = 3$
	\item $v\ 8\ 7 = 14$
	\item $v\ 7\ 9 = 14$
	\item $v\ 2 = 4$
	\item $v\ 9 = 18$
	\item $v\ 5 = 11$
	\item $v\ 2\ 6 = 9$
	\item $v\ 3\ 5 = 4$
	\item $v\ 8\ 2\ 5 = 7$
	\item $v\ 7\ 3\ 6 = 3$
	\item $v\ 8\ 1 = 2$
	\item $v\ 8\ 6 = 12$
	\item $v\ 4\ 9 = 8$
\end{enumerate}
\end{document}
