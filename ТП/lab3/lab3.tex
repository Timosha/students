\documentclass[a4paper,12pt]{article}

\usepackage[T2A]{fontenc}
\usepackage[utf8x]{inputenc}
\usepackage[russian,english]{babel}
\usepackage{graphicx}
\usepackage{color}
\usepackage{xcolor}
\usepackage{listings}
\usepackage{fancyhdr}
\usepackage{amsmath}
\usepackage{indentfirst} % включить отступ у первого абзаца

%%\usepackage[
%%		a4paper, includefoot,
%%		left=3cm, right=1cm, top=2cm, bottom=1.5cm,
%%		headsep=1cm, footskip=1cm
%%	]{geometry}

\title{Лабораторная работа №3}
\author{Вагин Д.А.}
\date{10/2010}
\pagestyle{empty}
\pagestyle{fancy}
\lhead{Лабораторная работа №3} %верхний колонтитул слева
\rhead{\title}


%%\lstset{language=c++,inputencoding=utf8x, extendedchars=\true,captionpos=b,tabsize=3,frame=lines,keywordstyle=\color{blue},commentstyle=\color{green},stringstyle=\color{red},numbers=left,numberstyle=\t	iny,numbersep=5pt,breaklines=true,showstringspaces=false,basicstyle=\footnotesize,emph={label}}

\lstloadlanguages{lisp}
\lstset{
	language=lisp,inputencoding=utf8x,
	extendedchars=\true,captionpos=b,tabsize=4,
	frame=lines,
	keywordstyle=\color{blue},commentstyle=\color{green},stringstyle=\color{red},
	breaklines=true,showstringspaces=false,basicstyle=\footnotesize
}

\begin{document}

%%\maketitle

\paragraph{Функции в LISP}
\subparagraph{Цели}
\begin{itemize}
	\item Познакомиться именованными фунциями
	\item Познакомиться с анонимными функциями
\end{itemize}

\paragraph{Задание}
\begin{enumerate}
	\item Написать функцию по первому заданию
	\item Написать функцию принимающую в качестве аргумента список заданного вида и возвращающую список такого же вида, но с изменёнными значениями.
	\item Написать анонимную функцию, которая передаётся парметром в функцию из второго задания, и выполняет некоторые действия над элементами списка.
\end{enumerate}

\paragraph{Пример}
\subparagraph{Задание}
\begin{enumerate}
	\item Функция принимает два аргумента и возвращает их сумму.
	\item Функция принимает список вида (x x x x x ... ) и увличивает каждый элемент списка на 1.
	\item Написать анонимную функцию которая преобразует каждый элемент в список. ((x) (x) (x)...)
\end{enumerate}

\begin{lstlisting}[language=lisp,{caption=Задание 1}]
(defun summa (a b) 
	(+ a b)
)

;; пример 
(summa 4 5)
\end{lstlisting}

\begin{lstlisting}[language=lisp,{caption=Задание 2}]
(defun mapx (x) 
	(if x
		(cons (+ (car x) 1) (mapx (cdr x)))
		nil
	)
)
;; пример вызова
(mapx '(1 2 3 4))
\end{lstlisting}

\begin{lstlisting}[language=lisp,{caption=Задание 3}]
(defun mapx (x f) 
	(if x
		(cons (funcall f (car x)) (mapx (cdr x) f))
		nil
	)
)
;; пример вызова
(mapx '(1 2 3 4) (lambda (x)(+ x 1)))
\end{lstlisting}

\subparagraph{Состав отчета}
\begin{itemize}
	\item Титульный лист (фамилия, группа, номер варианта, наименование работы, задание)
	\item Текст рекурсивной функции
	\item Результаты выполнения
\end{itemize}

\paragraph{Варианты заданий}
\begin{enumerate}
	\item \begin{enumerate}
		\item Функция принимает два числа, и если их сумма чётна, то возвращает их разницу, иначе - сумму.
		\item Дан список ((x x) (x x) (x x) ... ). Увеличить каждый элемент на единицу.
		\item Обеденить каждый подсписок суммированием: ((1 2) (1 3) (3 4)) -> ((3) (4) (7))
	\end{enumerate}

	\item \begin{enumerate}
		\item Функция принимает два числа и возвращает наибольшее из них
		\item Дан список ((x x x x ...) (x x x x ...)). Увеличить каждый элемент на единицу
		\item Умножить каждый элемент на произвольное число
	\end{enumerate}

	\item \begin{enumerate}
		\item Функция принимает 3 числа и возвращает список с этими числами
		\item Дан список (x (x (x (x...)))). Увеличить каждый элемент на единицу
		\item Умножить каждый элемент на 2
	\end{enumerate}

	\item \begin{enumerate}
		\item Функция принимает 1 число и возвращает квадрат этого числа если оно чётное, и куб, если нечётное
		\item Дан список ((((... x) x) x) x). Увеличить каждый элемент на единицу
		\item Поделить каждый элемент на 2
	\end{enumerate}

	\item \begin{enumerate}
		\item Функция принимает 2 числа и возвращает их произведение, если их сумма чётна, и квадрат первого, если сумма нечётна
		\item Дан список ((x (x (x))) (x (x (x))) (x (x (x))) ... ). Увеличить каждый элемент на единицу
		\item Увеличить каждый элемент в 2 раза
	\end{enumerate}

	\item \begin{enumerate}
		\item Функция принимает список и число, и добавляет число к списку.
		\item Дан список ((x x x) (x x x) ... ). Увеличить каждый элемент на единицу.
		\item Обеденить каждый подсписок суммированием: ((1 2 1) (1 3 2) (3 4 1)) -> ((4) (6) (8))
	\end{enumerate}

	\item \begin{enumerate}
		\item Функция принимает список 3 числа и возвращает список вида (x (x (x)))
		\item Дан список ((x) (x) ... ). Увеличить каждый элемент на единицу.
		\item Преобразовать список в простой список элементов: ((1) (3) (4)) -> (1 3 4)
	\end{enumerate}

	\item \begin{enumerate}
		\item Функция принимает список 3 числа и возвращает список вида (((x) x) x)
		\item Дан список ((x (x x)) (x (x x)) ... ). Увеличить каждый элемент на единицу.
		\item Увеличить каждый элемент в произвольное число раз
	\end{enumerate}

	\item \begin{enumerate}
		\item Функция принимает список 3 числа и возвращает список вида ((x) x (x))
		\item Дан список (((x) x (x)) ((x) x (x)) ... ). Увеличить каждый элемент на единицу.
		\item Увеличить каждый элемент в произвольное число раз
	\end{enumerate}

	\item \begin{enumerate}
		\item Функция принимает список 3 числа и возвращает список вида (x (x) x)
		\item Дан список (((x) (x)) ((x) (x)) ... ). Увеличить каждый элемент на единицу.
		\item Увеличить каждый элемент в произвольное число раз
	\end{enumerate}

	\item \begin{enumerate}
		\item Функция принимает список 3 числа и возвращает список вида ((x) x (x))
		\item Дан список ((x) (x) ... ). Увеличить каждый элемент на единицу.
		\item Увеличить каждый элемент в произвольное число раз
	\end{enumerate}

	\item \begin{enumerate}
		\item Функция принимает два числа и возвращает наименьшее из них
		\item Дан список ((((... x) x) x) x). Увеличить каждый элемент на 2.
		\item Увеличить каждый элемент в произвольное число раз
	\end{enumerate}

	\item \begin{enumerate}
		\item Функция принимает список и число, и добавляет число к списку с конца.
		\item Дан список ((x) (x) ... ). Увеличить каждый элемент на единицу.
		\item Поделить каждый элемент на 2
	\end{enumerate}

	\item \begin{enumerate}
		\item Функция принимает 3 числа и возвращает список с этими числами
		\item Дан список ((x) (x) ... ). Увеличить каждый элемент на единицу.
		\item Умножить каждый чётный элемент на произвольное число
	\end{enumerate}

	\item \begin{enumerate}
		\item Функция принимает два числа и возвращает первое, если оно кратно второму, и второе, если не кратно.
		\item Дан список ((x x x x ...) (x x x x ...)). Увеличить каждый элемент на единицу
		\item Обеденить каждый подсписок суммированием: ((1 2 1...) (1 3 2 ...) (3 4 1 ...)) -> ((4) (6) (8))
	\end{enumerate}

\end{enumerate}
\end{document}
