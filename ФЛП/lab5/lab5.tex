\documentclass[a4paper,12pt]{article}

\usepackage[T2A]{fontenc}
\usepackage[utf8x]{inputenc}
\usepackage[russian,english]{babel}
\usepackage{graphicx}
\usepackage{color}
\usepackage{xcolor}
\usepackage{listings}
\usepackage{fancyhdr}
\usepackage{amsmath}
\usepackage{textcomp}
\usepackage{amssymb}
\usepackage{indentfirst} % включить отступ у первого абзаца

%%\usepackage[
%%		a4paper, includefoot,
%%		left=3cm, right=1cm, top=2cm, bottom=1.5cm,
%%		headsep=1cm, footskip=1cm
%%	]{geometry}

\title{Лабораторная работа №5}
\author{Вагин Д.А.}
\date{03/2011}
\pagestyle{empty}
\pagestyle{fancy}
\lhead{Лабораторная работа №5} %верхний колонтитул слева
\rhead{\title}


%%\lstset{language=c++,inputencoding=utf8x, extendedchars=\true,captionpos=b,tabsize=3,frame=lines,keywordstyle=\color{blue},commentstyle=\color{green},stringstyle=\color{red},numbers=left,numberstyle=\t	iny,numbersep=5pt,breaklines=true,showstringspaces=false,basicstyle=\footnotesize,emph={label}}

\lstloadlanguages{haskell}
\lstset{
	language=haskell,inputencoding=utf8x,
	extendedchars=\true,captionpos=b,tabsize=4,
	frame=lines,
	keywordstyle=\color{blue},commentstyle=\color{green},stringstyle=\color{red},
	breaklines=true,showstringspaces=false,basicstyle=\footnotesize
}

\begin{document}

%%\maketitle

\paragraph{Основные возможности Haskell}
\subparagraph{Цели}
\begin{itemize}
	\item Приобрести навыки работы с интерпретатором языка Haskell. Получить представление об основных типах языка Haskell. Научиться определять простейшие функции.
\end{itemize}

\paragraph{Задание}
\begin{enumerate}
	\item Изучить теоретические сведения
	\item Выполнить задания в соответствии с вариантом
\end{enumerate}

\paragraph{Теоретические сведения}
\subparagraph{Типы}
\begin{itemize}
	\item Типы Integer и Int используется для представления целых чисел, причем значения типа Integer не ограничены по длине.
	\item Типы Float и Double используется для представления вещественных чисел.
	\item Тип Bool содержит два значения: True и False, и предназначен для представления результата логических выражений.
	\item  Тип Char используется для представления символов.
\end{itemize}

\subparagraph{Списки}
Чтобы задать список в Haskell, необходимо в квадратных скобках перечислить его элементы через запятую. Все эти элементы должны принадлежать одному и тому же типу. Тип списка с элементами, принадлежащими типу a, обозначается как [a]. Примеры: [1,2]; ['1','2','3'].

\subparagraph{Операции со списками}
Оператор : (двоеточие) используется для добавления элемента в начало списка. Его левым аргументом должен быть элемент, а правым - список:
\begin{lstlisting}[language=haskell,{caption=Пример}]
> 1:[2,3]
[1,2,3] :: [Integer]
> '5':['1','2','3','4','5']
['5','1','2','3','4','5'] :: [Char]
> False:[]
[False] :: [Bool]
\end{lstlisting}

С помощью оператора (:) и пустого списка можно построить любой список:
\begin{lstlisting}[language=haskell,{caption=Пример}]
> 1:(2:(3:[]))
[1,2,3] :: Integer
>1:2:3:[]
[1,2,3] :: Integer
\end{lstlisting}

Элементами списка могут быть любые значения — числа, символы, кортежи, другие списки и т.д.
\begin{lstlisting}[language=haskell,{caption=Пример}]
> [(1,'a'),(2,'b')]
[(1,'a'),(2,'b')] :: [(Integer,Char)]
> [[1,2],[3,4,5]]
[[1,2],[3,4,5]] :: [[Integer]]
\end{lstlisting}

Для работы со списками в языке Haskell существует большое количество функций. В данной лабораторной работе рассмотрим только некоторые из них.
\begin{itemize}
	\item Функция head возвращает первый элемент списка.
	\item Функция tail возвращает список без первого элемента.
	\item Функция length возвращает длину списка.
\end{itemize}
Функции head и tail определены для непустых списков. При попытке применить их к пустому списку интерпретатор сообщает об ошибке. Примеры работы с указанными функциями:

\begin{lstlisting}[language=haskell,{caption=Пример}]
> head [1,2,3]
1 :: Integer
> tail [1,2,3]
[2,3] :: [Integer]
> tail [1]
[] :: Integer
> length [1,2,3]
3 :: Int
\end{lstlisting}

Для соединения (конкатенации) списков в Haskell определен оператор ++.
\begin{lstlisting}[language=haskell,{caption=Пример}]
> [1,2]++[3,4]
[1,2,3,4] :: Integer
\end{lstlisting}

\subparagraph{Функции} Рассмотрим пример:

\begin{lstlisting}[language=haskell,{caption=Пример}]
square :: Integer -> Integer
square x = x * x
\end{lstlisting}
Первая строка (square :: Integer -> Integer) объявляет, что мы определяем функцию square, принимающую параметр типа Integer и возвращающую результат типа Integer. Вторая строка (square x = x * x) является непосредственно определением функции. Функция square принимает один аргумент и возвращает его квадрат.

В общем виде тип функции, принимающей n аргументов, принадлежащих типам t1, t2, . . . , tn, и возвращающей результат типа a, записывается в виде t1->t2->...->tn->a.
\begin{lstlisting}[language=haskell,{caption=Пример}]
add :: Integer -> Integer -> Integer
add x y = x + y
\end{lstlisting}

\subparagraph{Условное выражение} В общем виде выглядит так: \\
if условие then выражение else выражение.

Функцию signum, вычисляющую знак переданного ей аргумента:
\begin{lstlisting}[language=haskell,{caption=Пример}]
signum :: Integer -> Integer
signum x = if x > 0
        then 1
        else if x < 0
                then -1
                else 0
\end{lstlisting}
Следует обратить внимание на отступы. Именно по отступам компилятор определяет к какому if'у отностится тот или иной else.

Условие в определении условного оператора представляет собой любое выражение типа Bool. Примером таких выражений могут служить сравнения. При сравнении можно использовать следующие операторы:
\begin{itemize}
	\item <, >, <=, >= - эти операторы имеют такой же смысл, как и в языке Си (меньше, больше, меньше или равно, больше или равно);
	\item == - оператор проверки на равенство;
	\item /= - оператор проверки на неравенство.
\end{itemize}

\paragraph{Пример}
\subparagraph{Задание}
Написать функцию на языке Haskell возвращающую знак переданного целого числа.
\begin{lstlisting}[language=haskell,{caption=Пример}]
signum :: Integer -> Integer
signum x = if x > 0
        then 1
        else if x < 0
                then -1
                else 0

main = print $ Prelude.signum 1
\end{lstlisting}

\subparagraph{Состав отчета}
\begin{itemize}
	\item Титульный лист (фамилия, группа, номер варианта, наименование работы, задание)
	\item Текст программы на языке Haskell
	\item Результат работы программы на языке Haskell
\end{itemize}

\paragraph{Варианты заданий}
\begin{enumerate}
	\item Функция max3, по трем целым возвращающая наибольшее из них.
	\item Функция min3, по трем целым возвращающая наименьшее из них.
	\item Функция sort2, по двум целым возвращающая пару, в которой наименьшее из них стоит на первом месте, а наибольшее - на втором.
	\item Функция bothTrue :: Bool -> Bool -> Bool, которая возвращает True тогда и только тогда, когда оба ее аргумента будут равны True. Не используйте при определении функции стандартные логический операции  (\&\& , || и т.п.).
	\item Функция     solve2::Double->Double->(Bool,Double), которая по двум числам, представляющим собой коэффициенты линейного уравнения ax + b = 0, возвращает пару, первый    элемент которой равен True, если решение существует и False в противном случае; при этом второй элемент равен либо значению корня, либо 0.0.
	\item Функция isParallel, возвращающая True, если два отрезка, концы которых задаются в аргументах функции, параллельны (или лежат на одной прямой). Например, значение выражения isParallel (1,1) (2,2) (2,0) (4,2) должно быть равно True, поскольку отрезки (1, 1) - (2, 2) и (2, 0) - (4, 2) параллельны.
	\item Функция isIncluded, аргументами которой служат параметры двух окружностей на плоскости (координаты центров и радиусы). Функция возвращает True, если вторая окружность целиком содержится внутри первой.
 	\item Функция isRectangular, принимающая в качестве параметров координаты трех точек на плоскости, и возвращающая True, если образуемый ими треугольник - прямоугольный.
	\item Функция isTriangle, определяющая, можно ли их отрезков с заданными длинами x, y и z построить треугольник.
	\item Функция isSorted, принимающая на вход три числа и возвращающая True, если они упорядочены по возрастанию или по убыванию.
	\item Функция принимает два числа, и если их сумма чётна, то возвращает их разницу, иначе - сумму.
	\item Функция принимает список и число, и добавляет число к списку.
	\item Функция принимает список и число, и добавляет число к списку с конца.
	\item Функция принимает 3 числа и возвращает список с этими числами
	\item Функиця принимает 2 числа, и возвращает наиболшее кратное двойке.
\end{enumerate}
\end{document}
