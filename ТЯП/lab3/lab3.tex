\documentclass[a4paper,12pt]{article}

\usepackage[T2A]{fontenc}
\usepackage[utf8x]{inputenc}
\usepackage[russian,english]{babel}
\usepackage{graphicx}
\usepackage{color}
\usepackage{xcolor}
\usepackage{listings}
\usepackage{fancyhdr}
\usepackage{amsmath}
\usepackage{indentfirst} % включить отступ у первого абзаца
\usepackage{url}

%%\usepackage[
%%		a4paper, includefoot,
%%		left=3cm, right=1cm, top=2cm, bottom=1.5cm,
%%		headsep=1cm, footskip=1cm
%%	]{geometry}

\title{Лабораторная работа №4}
\author{Кузнецов Д.Б.\and Вагин Д.А.}
\date{10/2010}
\pagestyle{empty}
\pagestyle{fancy}
\lhead{Лабораторная работа №4} %верхний колонтитул слева

\lstloadlanguages{c++,html,sql,pascal}
\lstset{
	language=html,inputencoding=utf8x,
	extendedchars=\true,captionpos=b,tabsize=4,
	frame=lines,
	keywordstyle=\color{blue},commentstyle=\color{green},stringstyle=\color{red},
	breaklines=true,showstringspaces=false,basicstyle=\footnotesize
}
\lstset{
	language=c++,inputencoding=utf8x,
	extendedchars=\true,captionpos=b,tabsize=4,
	frame=lines,
	keywordstyle=\color{blue},commentstyle=\color{green},stringstyle=\color{red},
	breaklines=true,showstringspaces=false,basicstyle=\footnotesize
}
\lstset{
	language=sql,inputencoding=utf8x,
	extendedchars=\true,captionpos=b,tabsize=4,
	frame=lines,
	keywordstyle=\color{blue},commentstyle=\color{green},stringstyle=\color{red},
	breaklines=true,showstringspaces=false,basicstyle=\footnotesize
}
\lstset{
	language=pascal,inputencoding=utf8x,
	extendedchars=\true,captionpos=b,tabsize=4,
	frame=lines,
	keywordstyle=\color{blue},commentstyle=\color{green},stringstyle=\color{red},
	breaklines=true,showstringspaces=false,basicstyle=\footnotesize
}


\begin{document}

\paragraph{Построение синтаксического анализатора на основании LL(1) грамматики}

\paragraph{Порядок выполнения}
\begin{enumerate}
	\item Построить LL(1) грамматику
	\item Определить множества выбора для каждого правила
	\item Изобразить низходящую схему разбора
	\item Разработать лексический анализатор на базе lex
	\item Построить МП-автомат по грамматике
	\item Реализовать МП-автомат
	\item Реализовать синтаксический анализ методом рекурсивного спуска	
\end{enumerate}

\subparagraph{Состав отчета}
\begin{itemize}
	\item Титульный лист (фамилия, группа, номер варианта, наименование работы, задание)
\end{itemize}

\paragraph{Варианты заданий}
В задании указано содержательное описание грамматики и простейший пример для облегчения понимания.
\begin{enumerate}
	\item конструкция for языка с++. 
\begin{lstlisting}[language=c++,{caption=for}]
for(
	a=0, b=76; 
	i<10 && b>0 ;
	++i, b=i+1
){ 
	b++;
	i++;
}
\end{lstlisting}

	\item конструкция if языка c++. 
\begin{lstlisting}[language=c++,{caption=if}]
if(a==0 && b<=0)
{
	a=b; b++;
}
else
{
	a++; 
	b+=2;
}
\end{lstlisting}

	\item конструкция switch языка c++. 
\begin{lstlisting}[language=c++,{caption=switch}]
switch(a){ 
	case 1: a=1; 
	case 3: 
	case 2: a++; break; 
	default: a=0;
}
\end{lstlisting}

	\item тэг img языка разметки HTML: 
\begin{lstlisting}[language=html,{caption=img}]
<img src="/images/a.png" width="20" height="30" 
	alt="картинка" />
\end{lstlisting}

	\item Объявление функции в c++:
\begin{lstlisting}[language=c++,{caption=func}]
int funcname(int a, char b, float c = 0.1);
\end{lstlisting}

	\item тэг table языка разметки HTML: 
\begin{lstlisting}[language=html,{caption=table}]
<table>
	<tr>
		<th>заголовок 1</th>
		<th>заголовок 2</th>
	</tr>
	<tr>
		<td>ячейка 1</td>
		<td>ячейка 2</td>
	</tr>
	<tr>
		<td>ячейка 3</td>
		<td>ячейка 4</td>
	</tr>
</table>
\end{lstlisting}

	\item тэг ul (ненумерованный список) языка разметки HTML: 
\begin{lstlisting}[language=html,{caption=ul}]
<ul>
	<li>item 1</li>
	<li>item 2</li>
	<li>item 3</li>
</ul>
\end{lstlisting}

\item конструкция while языка c++. 
\begin{lstlisting}[language=c++,{caption=while}]
while( a>0 || b<76){
	b++;
	a++;
}
\end{lstlisting}

\item конструкция do-while языка c++. 
\begin{lstlisting}[language=c++,{caption=do-while}]
do{
	b++;
	a++;
}while( a>0 || b<76);
\end{lstlisting}

	\item конструкция insert языка запросов SQL:
\begin{lstlisting}[language=sql,{caption=insert}]
insert into tablename(pk,column1,column2) 
values (1, 2 , 'varchar');
\end{lstlisting}

	\item конструкция select языка запросов SQL:
\begin{lstlisting}[language=sql,{caption=select}]
select pk, column1, column2
from tablename
where column1 = 2 
	and column2 like 'xxx%';
\end{lstlisting}

	\item конструкция delete языка запросов SQL:
\begin{lstlisting}[language=sql,{caption=delete}]
delete from tablename
where column1 = 1
	or column2 like 'xxx%';
\end{lstlisting}

	\item тэг form языка разметки HTML: 
\begin{lstlisting}[language=html,{caption=form}]
<form methid="post" action="/registration/">
	<input type="text" name="login" />
	<input type="text" name="email" />
	<input type="password" name="password" />
	<input type="submit"/>
</form>
\end{lstlisting}

	\item конструкция if языка Pascal: 
\begin{lstlisting}[language=pascal,{caption=if}]
if (a <= 10) or (b >= 2) then
begin
	a := 10;
	b := 10;
end else
	a := b;

\end{lstlisting}

	\item конструкция for языка Pascal: 
\begin{lstlisting}[language=pascal,{caption=for}]
for i:= 0 to 10 step 2 do 
begin
	a := i;
end;
\end{lstlisting}

\end{enumerate}

\paragraph{Пример реализации}
\begin{itemize}
	\item Пример МП автомата: \url{http://www.softcraft.ru/translat/lect/t07-07.shtml}
	\item Пример программы с использованием рекурсивного спуска: \url{http://www.softcraft.ru/translat/lect/t07-08.shtml}
\end{itemize}


\end{document}
