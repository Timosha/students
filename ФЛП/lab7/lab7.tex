\documentclass[a4paper,12pt]{article}

\usepackage[T2A]{fontenc}
\usepackage[utf8x]{inputenc}
\usepackage[russian,english]{babel}
\usepackage{graphicx}
\usepackage{color}
\usepackage{xcolor}
\usepackage{listings}
\usepackage{fancyhdr}
\usepackage{amsmath}
\usepackage{textcomp}
\usepackage{amssymb}
\usepackage{indentfirst} % включить отступ у первого абзаца

%%\usepackage[
%%		a4paper, includefoot,
%%		left=3cm, right=1cm, top=2cm, bottom=1.5cm,
%%		headsep=1cm, footskip=1cm
%%	]{geometry}

\title{Лабораторная работа №7}
\author{Вагин Д.А.}
\date{05/2011}
\pagestyle{empty}
\pagestyle{fancy}
\lhead{Лабораторная работа №7} %верхний колонтитул слева
\rhead{\title}

\lstloadlanguages{prolog}
\lstset{
	language=prolog,inputencoding=utf8x,
	extendedchars=\true,captionpos=b,tabsize=4,
	frame=lines,
	keywordstyle=\color{blue},commentstyle=\color{green},stringstyle=\color{red},
	breaklines=true,showstringspaces=false,basicstyle=\footnotesize
}

\begin{document}

%%\maketitle

\paragraph{Формализация предметной области для языка Пролог}
\subparagraph{Цели}
\begin{itemize}
	\item Познакомиться языком Пролог
\end{itemize}

\paragraph{Задание}
\begin{enumerate}
	\item Для соответствующей варианту предметной области составить 3-5 аксиом.
	\item Составить несколько вопросов.
	\item Записать на языке Пролог.
\end{enumerate}

\paragraph{Пример}
\subparagraph{Задание}
\begin{itemize}
	\item Предметная область: раковина;
	\item описание предметной области: Раковина засоряется тем, что пропустит фильтр. Фильтр пропускает очистки морковки при использовании любых инструментов для чистки и очистки любого овощя очищенного картофелечисткой. Морковь почистили ножом, картофель - картофелечисткой;
	\item вопросы: Чем засорится раковина? Засорится ли раковина?.
\end{itemize}

\subparagraph{Решение}
Объявим предикаты
\begin{itemize}
	\item $C(x,y)$ - чистить икс игреком
	\item $F(x)$ - фильтр пропустит х
	\item $R(x)$ - раковина засорилась х
\end{itemize}

\begin{align*}
A1&: F(x) \to R(x)\\
A2&: C(x, kartochistka) \to F(x)\\
A3&: C(morkva, y) \to F(morkva)\\
A4&: C(morkva, nozh)\\
A5&: C(kartofan, kartochistka)\\
B1&: R(x)\\
B2&: R(kartofan)\\
\end{align*}

Напишем программу на языке Пролог:
\begin{lstlisting}[language=prolog,{caption=Пример}]
r(X) :- f(X).
f(X) :- c(X, 'kartochistka').
f('morkva') :- c('morkva', _).
c('morkva', 'nozh').
c('kartofan', 'kartochistka').
\end{lstlisting}

Зададим пару вопросов программе:
\begin{lstlisting}[language=prolog,{caption=Результат}]
$ gprolog
GNU Prolog 1.2.18
By Daniel Diaz
Copyright (C) 1999-2004 Daniel Diaz
| ?- consult('rack.pl').
compiling /home/kdb/prolog/rack.pl for byte code...
/home/kdb/prolog/rack.pl compiled, 5 lines read - 739 bytes written, 25 ms

yes
| ?- r('morkva').

yes
| ?- r(X).

X = kartofan ? a

X = morkva

yes
| ?- trace.
The debugger will first creep -- showing everything (trace)

yes

{trace}
| ?- r('kartofan').
      1    1  Call: r(kartofan) ? 
      2    2  Call: f(kartofan) ? 
      3    3  Call: c(kartofan,kartochistka) ? 
      3    3  Exit: c(kartofan,kartochistka) ? 
      2    2  Exit: f(kartofan) ? 
      1    1  Exit: r(kartofan) ? 

yes
{trace}
| ?- r(X).         
      1    1  Call: r(_16) ? 
      2    2  Call: f(_16) ? 
      3    3  Call: c(_16,kartochistka) ? 
      3    3  Exit: c(kartofan,kartochistka) ? 
      2    2  Exit: f(kartofan) ? 
      1    1  Exit: r(kartofan) ? 

X = kartofan ?  a
      1    1  Redo: r(kartofan) ? 
      2    2  Redo: f(kartofan) ? 
      3    3  Call: c(morkva,_69) ? 
      3    3  Exit: c(morkva,nozh) ? 
      2    2  Exit: f(morkva) ? 
      1    1  Exit: r(morkva) ? 

X = morkva

yes
{trace}
| ?- 
\end{lstlisting}

\subparagraph{Состав отчета}
\begin{itemize}
	\item Титульный лист (фамилия, группа, номер варианта, наименование работы, задание)
	\item Задание
	\item Аксиомы
	\item Вопросы
	\item Программа на языке Пролог
\end{itemize}

\paragraph{Варианты заданий}
\begin{enumerate}
	\item книжный шкаф
	\item зоопарк
	\item театр
	\item цирк
	\item телевизор
	\item баня
	\item таблица умножения
	\item сдача экзамена
	\item ноутбук
	\item файловая система
	\item правительства
	\item система прав пользователей
	\item продуктовый магазин
	\item учебная аудитория
	\item родственники
	\item структура каталогов
	\item арифметические действия
	\item текст
	\item гипертекст
	\item холодильник
	\item лес
	\item огород
\end{enumerate}
\end{document}
