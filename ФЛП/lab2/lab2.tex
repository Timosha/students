\documentclass[a4paper,12pt]{article}

\usepackage[T2A]{fontenc}
\usepackage[utf8x]{inputenc}
\usepackage[russian,english]{babel}
\usepackage{graphicx}
\usepackage{color}
\usepackage{xcolor}
\usepackage{listings}
\usepackage{fancyhdr}
\usepackage{amsmath}
\usepackage{indentfirst} % включить отступ у первого абзаца

%%\usepackage[
%%		a4paper, includefoot,
%%		left=3cm, right=1cm, top=2cm, bottom=1.5cm,
%%		headsep=1cm, footskip=1cm
%%	]{geometry}

\title{Лабораторная работа №2}
\author{Вагин Д.А.}
\date{02/2011}
\pagestyle{empty}
\pagestyle{fancy}
\lhead{Лабораторная работа №2} %верхний колонтитул слева
\rhead{\title}


%%\lstset{language=c++,inputencoding=utf8x, extendedchars=\true,captionpos=b,tabsize=3,frame=lines,keywordstyle=\color{blue},commentstyle=\color{green},stringstyle=\color{red},numbers=left,numberstyle=\t	iny,numbersep=5pt,breaklines=true,showstringspaces=false,basicstyle=\footnotesize,emph={label}}

\lstloadlanguages{lisp}
\lstset{
	language=lisp,inputencoding=utf8x,
	extendedchars=\true,captionpos=b,tabsize=4,
	frame=lines,
	keywordstyle=\color{blue},commentstyle=\color{green},stringstyle=\color{red},
	breaklines=true,showstringspaces=false,basicstyle=\footnotesize
}

\begin{document}

%%\maketitle

\paragraph{S-выражения в LISP}
\subparagraph{Цели}
\begin{itemize}
	\item Освоить S-выражения
	\item Научиться основам работы в clisp
	\item Познакомиться с функциями обработки списков
\end{itemize}

\paragraph{Задание}
\begin{enumerate}
	\item Составить список по заданию в синтаксисе lisp
	\item Написать функции для получения каждого из элементов списка №1
	\item Написать функцию для получения списка №2
\end{enumerate}

\paragraph{Пример}
\subparagraph{Задание}
\begin{enumerate}
	\item Список №1 - \{\{1,\{2,3,4,5,6,7,8\}\},9\}
	\item Список №2 - \{2,8,3,\{4,1\},6\}
\end{enumerate}

\begin{lstlisting}[language=lisp,{caption=Задание 1}]
'((1 (2 3 4 5 6 7 8)) 9)
\end{lstlisting}

\begin{lstlisting}[language=lisp,{caption=Задание 2}]
(car '((1 (2 3 4 5 6 7 8)) 9))
;; (1 (2 3 4 5 6 7 8))
(car (car '((1 (2 3 4 5 6 7 8)) 9)))
;; 1
(car (cdr (car '((1 (2 3 4 5 6 7 8)) 9))))
;; (2 3 4 5 6 7 8)
(car (car (cdr (car '((1 (2 3 4 5 6 7 8)) 9)))))
;; 2
(setq a '((1 (2 3 4 5 6 7 8)) 9))
;; ((1 (2 3 4 5 6 7 8)) 9)
(car (car (cdr (car a))))
;; 2
(car (cdr (car (cdr (car a)))))
;; 3
(car (cdr (cdr (car (cdr (car a))))))
;; ...
\end{lstlisting}

\begin{lstlisting}[language=lisp,{caption=Задание 3}]
(cons 2	(cons 8 (cons 3 (cons (cons 4 (cons 1 nil)) (cons 6 nil)))))
;;(2 8 3 (4 1) 6)
\end{lstlisting}

\subparagraph{Состав отчета}
\begin{itemize}
	\item Титульный лист (фамилия, группа, номер варианта, наименование работы, задание)
	\item Текст рекурсивной функции
	\item Результаты выполнения
\end{itemize}

\paragraph{Варианты заданий}
\begin{enumerate}
	\item \begin{enumerate}
		\item Список №1 - \{1,\{2,3,4\},5,\{6,\{7,8\},9\}\}
		\item Список №2 - \{2,8,\{3,4,1\},6\}
	\end{enumerate}

	\item \begin{enumerate}
		\item Список №1 - \{\{1,2,\{3,4,5,6\},7\},8,9\}
		\item Список №2 - \{2,\{8,3,4,1\},6\}
	\end{enumerate}

	\item \begin{enumerate}
		\item Список №1 - \{\{\{1,2,3,4\},5,6,7,8\},9\}
		\item Список №2 - \{\{7,8\},\{3,4\},\{1,6\}\}
	\end{enumerate}

	\item \begin{enumerate}
		\item Список №1 - \{1,2,\{3,4,5\},\{\{6,7,8,9\}\}\}
		\item Список №2 - \{\{2,8\},6,7,8,9\}
	\end{enumerate}

	\item \begin{enumerate}
		\item Список №1 - \{1,2,\{3,4,5\},\{6,7,\{8,9\}\}\}
		\item Список №2 - \{2,8,\{3,4,5\},6\}
	\end{enumerate}

	\item \begin{enumerate}
		\item Список №1 - \{1,\{\{2,3,4\},5,6,7\},8,9\} 
		\item Список №2 - \{2,\{\{8\},3,4,1,6\}\}
	\end{enumerate}

	\item \begin{enumerate}
		\item Список №1 - \{1,\{2,3,\{4,5,6\},7,8\},9\}
		\item Список №2 - \{2,\{8,3,4,\{1\}\},6\}
	\end{enumerate}

	\item \begin{enumerate}
		\item Список №1 - \{\{\{1,2,3,4,\}5,6,7,8\},9\}
		\item Список №2 - \{2,8,3,\{4,1\},6\}
	\end{enumerate}

	\item \begin{enumerate}
		\item Список №1 - \{\{\{1,2,3,4,5,6,7,8\},9\}\}
		\item Список №2 - \{2,8,3,4,\{1,6\}\}
	\end{enumerate}

	\item \begin{enumerate}
		\item Список №1 - \{1,2,3,\{4,\{5,6,\{7,8,9\}\}\}\}
		\item Список №2 - \{2,\{8,3\},4,1,6\}
	\end{enumerate}

	\item \begin{enumerate}
		\item Список №1 - \{1,\{2,3,\{4,5,6,\{7,8\},9\}\}\}
		\item Список №2 - \{2,8,3,4,\{\{1\},6\}\}
	\end{enumerate}

	\item \begin{enumerate}
		\item Список №1 - \{1,\{\{2,3\},4,5,6\},7,8,9\}
		\item Список №2 - \{\{2\},8,\{3,\{4,1\}\},6\}
	\end{enumerate}

	\item \begin{enumerate}
		\item Список №1 - \{1,\{2,\{3\},4,5\},6,\{7,8\},9\}
		\item Список №2 - \{2,\{\{8,3\},\{4,1\}\},6\}
	\end{enumerate}

	\item \begin{enumerate}
		\item Список №1 - \{\{1,\{2\},3,4\},5,6,\{7,\{8,9\}\}\}
		\item Список №2 - \{2,\{8,3,\{4\},1\},\{6\}\}
	\end{enumerate}

	\item \begin{enumerate}
		\item Список №1 - \{1,\{2,3,\{4,5,\{6,\{7\}\}\},8\},9\}
		\item Список №2 - \{2,\{8,\{3,\{4\}\},1\},6\}
	\end{enumerate}

	\item \begin{enumerate}
		\item Список №1 - \{\{1\},2,\{3,4\},\{5,6,\{7,8\},9\}\}
		\item Список №2 - \{\{2,8\},\{3,\{4,1\},6\}\}
	\end{enumerate}

	\item \begin{enumerate}
		\item Список №1 - \{1,\{2,3\},4,\{5\},6,7,8,9\}
		\item Список №2 - \{2,\{8\},\{3,4,\{1\}\},6\}
	\end{enumerate}

	\item \begin{enumerate}
		\item Список №1 - \{1,2,\{3\},\{4\},\{5\},6,\{7,\{8\}\},9\}
		\item Список №2 - \{\{2,\{8,3\},4,1\},\{6\}\}
	\end{enumerate}

	\item \begin{enumerate}
		\item Список №1 - \{1,2,\{3,\{4\},5\},\{6,\{7\},8\},9\}
		\item Список №2 - \{\{2,\{8,3,4,\{1\}\}\},6\}
	\end{enumerate}

	\item \begin{enumerate}
		\item Список №1 - \{\{1,2\},\{\{3,\{4\},5\},\{6,\{7\},8\}\},\{9\}\}
		\item Список №2 - \{2,\{\{8\},3,\{4,1\}\},\{6\}\}
	\end{enumerate}

\end{enumerate}
\end{document}
