\documentclass[a4paper,12pt]{article}

\usepackage[T2A]{fontenc}
\usepackage[utf8x]{inputenc}
\usepackage[russian,english]{babel}
\usepackage{graphicx}
\usepackage{color}
\usepackage{xcolor}
\usepackage{listings}
\usepackage{fancyhdr}
\usepackage{amsmath}
\usepackage{textcomp}
\usepackage{amssymb}
\usepackage{indentfirst} % включить отступ у первого абзаца

%%\usepackage[
%%		a4paper, includefoot,
%%		left=3cm, right=1cm, top=2cm, bottom=1.5cm,
%%		headsep=1cm, footskip=1cm
%%	]{geometry}

\title{Лабораторная работа №4}
\author{Вагин Д.А.}
\date{03/2011}
\pagestyle{empty}
\pagestyle{fancy}
\lhead{Лабораторная работа №4} %верхний колонтитул слева
\rhead{\title}


%%\lstset{language=c++,inputencoding=utf8x, extendedchars=\true,captionpos=b,tabsize=3,frame=lines,keywordstyle=\color{blue},commentstyle=\color{green},stringstyle=\color{red},numbers=left,numberstyle=\t	iny,numbersep=5pt,breaklines=true,showstringspaces=false,basicstyle=\footnotesize,emph={label}}

\lstloadlanguages{lisp}
\lstset{
	language=lisp,inputencoding=utf8x,
	extendedchars=\true,captionpos=b,tabsize=4,
	frame=lines,
	keywordstyle=\color{blue},commentstyle=\color{green},stringstyle=\color{red},
	breaklines=true,showstringspaces=false,basicstyle=\footnotesize
}

\begin{document}

%%\maketitle

\paragraph{$\lambda$-выражения и $\beta$-редукция в $\lambda$-исчислении в языке LISP}
\subparagraph{Цели}
\begin{itemize}
	\item Постигнуть $\lambda$-исчисление
\end{itemize}

\paragraph{Задание}
\begin{enumerate}
	\item Выполнить $\beta$-редукции несколькими способами
	\item Составить программу на LISP для вычисления функции
\end{enumerate}

\paragraph{Пример}
\subparagraph{Задание}
\[((((\lambda xyz.x(yz))(\lambda x.x\cdot 1))(\lambda x.x+x))2)\]

\begin{lstlisting}[language=lisp,{caption=Программа на LISP}]
( 
  (lambda (x y z) (funcall x (funcall y z))) 
  (lambda (x) (* x 1)) 
  (lambda (x) (+ x x)) 
2)

;; результат
4
\end{lstlisting}

\begin{align*}
(((\lambda xyz.x(yz))(\lambda x.x))(\lambda x.x+x))2=\\
=((\lambda yz.(\lambda x.x)(yz))(\lambda x.x+x))2 =\\
=(\lambda z.(\lambda x.x)((\lambda x.x+x)z))2
\end{align*}

\[(\lambda x.x)((\lambda x.x+x)2)=(\lambda x.x+x)2=2+2=4\]
\[(\lambda z.((\lambda x.x+x)z))2=(\lambda z.z+z)2=2+2=4\]


\subparagraph{Состав отчета}
\begin{itemize}
	\item Титульный лист (фамилия, группа, номер варианта, наименование работы, задание)
	\item Текст LISP программы
	\item $\beta$-редукции
\end{itemize}

\paragraph{Варианты заданий}
\begin{enumerate}
	\item \[ ((((\lambda xyz.x(yz))(\lambda x.x+x))(\lambda x.x\cdot x))3) \]
	\item \[ ((((\lambda xyz.x(yz))(\lambda x.x\cdot x))(\lambda x.x+x))4) \]
  \item \[ ((((\lambda xyz.x(yz))(\lambda x.x\cdot x))(\lambda x.x\cdot x))5) \]
	\item \[ ((((\lambda xyz.x(yz))(\lambda x.x+8))(\lambda x.9\cdot x))6) \]
	\item \[ ((((\lambda xyz.x(yz))(\lambda x.x))(\lambda x.x\cdot x))7)\]
	\item \[ ((((\lambda xyz.x(yz))(\lambda x.x+x))(\lambda x.x+x))8)\]
	\item \[ (((((\lambda xyz.xzy))(\lambda xy.x\div y))3)9)\]
	\item \[ (((((\lambda xyz.xzy))(\lambda xy.x\div y))((\lambda x.x)3))9)\]
	\item \[ (((((\lambda xyz.xzy))(\lambda xy.x\div y))3)((\lambda x.x)9))\]
	\item \[ ((((\lambda xyz.xz(yz)))(\lambda xy.x+y))(\lambda x.x+2)3)\]
	\item \[ ((((\lambda xyz.xz(yz)))(\lambda xy.x))(\lambda x.x)4))\]
	\item \[ ((((\lambda xyz.xz(yz)))(\lambda xy.y))(\lambda x.x+2)5)\]
	\item \[ ((((\lambda xyz.xz(yz)))(\lambda xy.x\cdot y))(\lambda x.x+2)6)\]
	\item \[ ((((\lambda xyz.xz(yz)))(\lambda xy.x+y))(\lambda x.x\cdot 2)7)\]
	\item \[ ((((\lambda xyz.xz(yz)))(\lambda xy.x-y))(\lambda x.x-2)8)\]
\end{enumerate}
\end{document}
