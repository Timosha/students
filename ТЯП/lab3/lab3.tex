\documentclass[a4paper,12pt]{article}

\usepackage[T2A]{fontenc}
\usepackage[utf8x]{inputenc}
\usepackage[russian,english]{babel}
\usepackage{graphicx}
\usepackage{color}
\usepackage{xcolor}
\usepackage{listings}
\usepackage{fancyhdr}
\usepackage{amsmath}
\usepackage{indentfirst} % включить отступ у первого абзаца
\usepackage{array}
\usepackage{supertabular}
\usepackage{hhline}
\usepackage{semantic}

%%\usepackage[
%%		a4paper, includefoot,
%%		left=3cm, right=1cm, top=2cm, bottom=1.5cm,
%%		headsep=1cm, footskip=1cm
%%	]{geometry}

\title{Лабораторная работа №3}
\author{Кузнецов Д.Б.\and Вагин Д.А.}
\date{10/2010}
\pagestyle{empty}
\pagestyle{fancy}
\lhead{Лабораторная работа №3} %верхний колонтитул слева

\lstloadlanguages{c++,c,tex,bash}
\lstset{
	language=tex,inputencoding=utf8x,
	extendedchars=\true,captionpos=b,tabsize=4,
	frame=lines,
	keywordstyle=\color{blue},commentstyle=\color{green},stringstyle=\color{red},
	breaklines=true,showstringspaces=false,basicstyle=\footnotesize
}

\begin{document}

\paragraph{Построение лексического анализатора с использованием lex}
\subparagraph{Цели}
\begin{itemize}
	\item Познакомиться с «регулярными выражениями»
	\item Познакомиться с программой lex
	\item Научиться строить лексические анализаторы с использованием lex
\end{itemize}

\paragraph{Порядок выполнения}
\begin{enumerate}
	\item Построить регулярное выражение
	\item Составить файл для lex
	\item Получить программу на Си
	\item Откомпилировать и запустить
	\item Написать отчет
\end{enumerate}

\subparagraph{Состав отчета}
\begin{itemize}
	\item Титульный лист (фамилия, группа, номер варианта, наименование работы, задание)
	\item Текст задания
	\item Текст программы на lex
	\item Результаты работы программы
\end{itemize}

\paragraph{Варианты заданий}
В задании указано содержательное описание грамматики и простейши пример для облегчения понимания.
\begin{enumerate}
	\item Одинаковые символы стоят парами: $aabbaabbbbaa$
	\item В начале строки $a$: $ababbbabb$
	\item Первый символ — не важен, далее одни $a$: $baaaaaaaaa$, $aaaaaaaaaa$
	\item Либо одни $a$, либо одни $b$: $aaaaaaaaaa$, $bbbbbbbbb$
	\item В конце строки $b$: $ababbbabb$
	\item Одинаковые символы не должны стоять рядом: $ababababab$, $bababababa$
	\item В строке должна встретиться хотя бы одна буква $a$: $bbbbbabbb$, $aaaaaaaaa$
	\item Предпоследним символом строки должна быть $b$, $abbabaabb$
	\item Вторым символом строки должна быть $a$: $baaaaabbb$
	\item Два последних символа должны быть $b$: $abababb$, $bbbbbb$
	\item Первый и третий символы должны быть разными: $aabbbabab$, $baaabbbab$
	\item Первый и последний символы должны быть одинковыми: $ababababa$, $babbbabab$
\end{enumerate}

\paragraph{Пример}
\subparagraph{Задание}
Символы a и b стоят парами: abbaabab, baabbaba.
\subparagraph{Регулярное выражение}
Регулярное выражение будет иметь вид: (ab|ba)+

\subparagraph{Программа на lex}
Создадим файл lab3.l со следующим содержимым:
\begin{lstlisting}[language=c,{caption=lab3.l}]
%%
(ab|ba)+ { printf("Yes");}
.* { printf("No"); }
%%

yyerror(char *str)
{ printf(str); }

main()
{ yylex(); }
\end{lstlisting}

\subparagraph{Получение программы на c}
Программу на c получим простой командой. lab.c - имя выходного файла, lab3.l имя входного файла на lex.
\begin{lstlisting}[language=bash,{caption=трансляция}]
flex -o lab3.c lab3.l
\end{lstlisting}

\subparagraph{Компиляция и запуск}
Для компиляции воспользуемся компилятором gcc.
\begin{lstlisting}[language=bash,{caption=компиляция и запуск}]
gcc -o lab3 lab3.c -lfl
./lab3

ababbababa
Yes

bbbaa
No	
\end{lstlisting}


\end{document}
